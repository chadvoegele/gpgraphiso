%% Template from: http://www.ieee.org/conferences_events/conferences/publishing/templates.html

\documentclass[conference]{IEEEtran}
\usepackage{graphicx}
\usepackage{amsmath}
\usepackage{algorithmic}
\usepackage{array}
\usepackage{url}

\begin{document}
\title{Using IrGL to Build Efficient Graph Programs}
\author{\IEEEauthorblockN{Chad Voegele}
\IEEEauthorblockA{University of Texas-Austin\\
chad@cs.utexas.edu}
\and
\IEEEauthorblockN{Sreepathi Pai}
\IEEEauthorblockA{University of Texas-Austin\\
sreepai@ices.utexas.edu}
\and
\IEEEauthorblockN{Yi-Shan Lu}
\IEEEauthorblockA{University of Texas-Austin\\
yishanlu@utexas.edu}}
\maketitle

% Descriptions from http://graphchallenge.mit.edu/submit
% A short summary of the work and results.
\begin{abstract}
The abstract goes here.
\end{abstract}

% Describe the practical graph analysis problems your contribution addresses and include a discussion and citations to how others in the field are addressing similar problems.
\section{Introduction}
intro
% static graph analysis
% irgl intermediate code easier to understand than sequences of matrix ops
% linear alegbra kernels as building blocks
% general subgraph isomorphism - gpsm, stwig,

% Describe in technical detail your specific contribution/innovation to graph analysis software, hardware, algorithm, and/or systems.
\section{Approach}
approach
% irgl based approach for high performance without barebones cuda coding
% triangle counting via edge intersection
% upper bound largest ktruss decomposition via largest k such that #edges with k is >= (k choose 3)
% run connected components to find largest component

% Describe those elements of the IEEE HPEC Graph Challenge that you selected to best highlight your contributions/innovations. Explain why you selected these elements and how you measured them for your implementation.
\section{Experiments}
expts

% Present tables and/or graphs presenting your measurements on your implementation and how they compare to the baseline implementation available at GraphChallenge.org.
\section{Results}
data
% correctness checks
% for both triangle and ktruss:
% runtimes of reference code vs irgl runtime

% A short summary of the work and potential next steps.
\section{Conclusion}
The conclusion goes here.

% use section* for acknowledgment
\section*{Acknowledgment}

The authors would like to thank...

\nocite{*}
\bibliographystyle{IEEEtran}
\bibliography{paper}

\end{document}
